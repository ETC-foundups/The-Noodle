%%%%%%%%%%%%%%%%%%%%%%%%%%%%%%%%%%%%%%%%%
% Journal Article
% LaTeX Template
% Version 1.4 (15/5/16)
%
% This template has been downloaded from:
% http://www.LaTeXTemplates.com
%
% Original author:
% Frits Wenneker (http://www.howtotex.com) with extensive modifications by
% Vel (vel@LaTeXTemplates.com)
%
% License:
% CC BY-NC-SA 3.0 (http://creativecommons.org/licenses/by-nc-sa/3.0/)
%
%%%%%%%%%%%%%%%%%%%%%%%%%%%%%%%%%%%%%%%%%

%----------------------------------------------------------------------------------------
%	PACKAGES AND OTHER DOCUMENT CONFIGURATIONS
%----------------------------------------------------------------------------------------

\documentclass[twoside,twocolumn]{article}

\usepackage{blindtext} % Package to generate dummy text throughout this template 
%\usepackage[utf8]{inputenc} % Package for unicode characters
\usepackage[utf8]{inputenc}
\usepackage{amssymb}
\usepackage{newunicodechar}
\newunicodechar{Ɖ}{\DH}

\usepackage[sc]{mathpazo} % Use the Palatino font
\usepackage[T1]{fontenc} % Use 8-bit encoding that has 256 glyphs
\linespread{1.05} % Line spacing - Palatino needs more space between lines
\usepackage{microtype} % Slightly tweak font spacing for aesthetics

\usepackage[english]{babel} % Language hyphenation and typographical rules

\usepackage[hmarginratio=1:1,top=32mm,columnsep=20pt]{geometry} % Document margins
\usepackage[hang, small,labelfont=bf,up,textfont=it,up]{caption} % Custom captions under/above floats in tables or figures
\usepackage{booktabs} % Horizontal rules in tables

\usepackage{lettrine} % The lettrine is the first enlarged letter at the beginning of the text

\usepackage{enumitem} % Customized lists
\setlist[itemize]{noitemsep} % Make itemize lists more compact

\usepackage{abstract} % Allows abstract customization
\renewcommand{\abstractnamefont}{\normalfont\bfseries} % Set the "Abstract" text to bold
\renewcommand{\abstracttextfont}{\normalfont\small\itshape} % Set the abstract itself to small italic text

\usepackage{titlesec} % Allows customization of titles
\renewcommand\thesection{\Roman{section}} % Roman numerals for the sections
\renewcommand\thesubsection{\roman{subsection}} % roman numerals for subsections
\titleformat{\section}[block]{\large\scshape\centering}{\thesection.}{1em}{} % Change the look of the section titles
\titleformat{\subsection}[block]{\large}{\thesubsection.}{1em}{} % Change the look of the section titles

\usepackage{fancyhdr} % Headers and footers
\pagestyle{fancy} % All pages have headers and footers
\fancyhead{} % Blank out the default header
\fancyfoot{} % Blank out the default footer
\fancyhead[C]{Ethereum Classic Library $\bullet$ October 2016 $\bullet$ Vol. I, No. 1} % Custom header text
\fancyfoot[RO,LE]{\thepage} % Custom footer text

\usepackage{titling} % Customizing the title section

\usepackage[pagebackref]{hyperref} % For hyperlinks in the PDF

%----------------------------------------------------------------------------------------
%	TITLE SECTION
%----------------------------------------------------------------------------------------
\setlength{\droptitle}{-4\baselineskip} % Move the title up

\pretitle{\begin{center}\Huge\bfseries} % Article title formatting
\posttitle{\end{center}} % Article title closing formatting
\title{The Noodle} % Article title
\author{%
\textsc{Prophet Daniel}\thanks{The author would like to thank the Ethereum Classic community.} \\[1ex] % Your name
\normalsize University of Nicosia \\ % Your institution
\normalsize \href{mailto:prophetdaniel@ethereumclassic.org}{prophetdaniel@ethereumclassic.org} % Your email address
%\and % Uncomment if 2 authors are required, duplicate these 4 lines if more
%\textsc{Jane Smith}\thanks{Corresponding author} \\[1ex] % Second author's name
%\normalsize University of Utah \\ % Second author's institution
%\normalsize \href{mailto:jane@smith.com}{jane@smith.com} % Second author's email address
}
\date{\today} % Leave empty to omit a date
\renewcommand{\maketitlehookd}{%
\begin{abstract}
\noindent In this paper a novel type of decentralized autonomous organization called Smart ƉAO is proposed to deal with increasingly complex problems utilizing artificial intelligence concepts on the smart contracts field.
\end{abstract}
}

%----------------------------------------------------------------------------------------

\begin{document}

% Print the title
\maketitle

%----------------------------------------------------------------------------------------
%	ARTICLE CONTENTS
%----------------------------------------------------------------------------------------

\section{Introduction}

\lettrine[nindent=0em,lines=3]{T}he Noodle is a set of intelligent agents working collectively in the context of a Decentralized Autonomous Organization (DAO). DAOs are implemented as a set of smart contracts, interacting among them to perform the desired functionality for the organization behavior. The state of the art in this field utilizes the Ethereum blockchain to deploy the aforementioned smart contracts. Unfortunately though, Ethereum still does not support intelligent agents development and deployment in its blockchain, what it does support is interacting with the outer world by means of its underlying blockchain transactions which can work as outgoing events or incoming events. As one of the main characteristics of blockchains is cloud computing and decentralization, it makes sense to utilize a similar characteristics platform for deploying The Noodle while Ethereum does not have one for that specific purpose.
%------------------------------------------------

\section{Cloud Computing}

Cloud computing is a type of Internet-based computing that provides shared computer processing resources and data to computers and other devices on demand. It is a model for enabling ubiquitous, on-demand access to a shared pool of configurable computing resources (e.g., computer networks, servers, storage, applications and services), which can be rapidly provisioned and released with minimal management effort.
Cloud infrastructure options comprehend:

\begin{itemize}
\item Software as a Service (SaaS)
\item Platform as a Service (PaaS)
\item Infrastructure as a Service (IaaS)
\end{itemize}

The designer of The Noodle has to take into consideration these options has to select best according to specific needs of the SmartDAO required functionality he is looking for.

\section{Intelligent Agents}

An artificial intelligent agent is programmed with AI tools to solve specific problems. As for example detecting the right ZIP numbers people manually write to letters utilizing a simple camera as source of information. This is a use case well known on the neural network field where results are surprisingly better than any human could reach.

\section{Conclusion}

Depending on the desired traits of the Smart DAO to be designed, its noodle will have to be modeled utilizing appropriate AI toolboxes to deliver the computing performance desired to reach the organization ultimate goal.

%Maecenas sed ultricies felis. Sed imperdiet dictum arcu a egestas. 
%\begin{itemize}
%\item Donec dolor arcu, rutrum id molestie in, viverra sed diam
%\item Curabitur feugiat
%\item turpis sed auctor facilisis
%\item arcu eros accumsan lorem, at posuere mi diam sit amet tortor
%\item Fusce fermentum, mi sit amet euismod rutrum
%\item sem lorem molestie diam, iaculis aliquet sapien tortor non nisi
%\item Pellentesque bibendum pretium aliquet
%\end{itemize}
%\blindtext % Dummy text

%Text requiring further explanation\footnote{Example footnote}.

%------------------------------------------------

%\section{Results}

%\begin{table}
%\caption{Example table}
%\centering
%\begin{tabular}{llr}
%\toprule
%\multicolumn{2}{c}{Name} \\
%\cmidrule(r){1-2}
%First name & Last Name & Grade \\
%\midrule
%John & Doe & $7.5$ \\
%Richard & Miles & $2$ \\
%\bottomrule
%\end{tabular}
%\end{table}

%\blindtext % Dummy text

%\begin{equation}
%\label{eq:emc}
%e = mc^2
%\end{equation}

%\blindtext % Dummy text

%------------------------------------------------

%\section{Discussion}

%\subsection{Subsection One}

%A statement requiring citation \cite{Figueredo2009}.
%\blindtext % Dummy text

%\subsection{Subsection Two}

%\blindtext % Dummy text

%----------------------------------------------------------------------------------------
%	REFERENCE LIST
%----------------------------------------------------------------------------------------

\bibliographystyle{abbrv}  
\bibliography{tex/bibliography}

%----------------------------------------------------------------------------------------

\end{document}
